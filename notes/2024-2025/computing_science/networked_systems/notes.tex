\documentclass{article}
\usepackage[utf8]{inputenc}
\usepackage{amsmath}
\usepackage{amsthm}
\usepackage[margin=2px]{geometry}

\newtheorem{definition}{Definition}
\newtheorem{note}{Note}

\title{Networked Systems}
\author{}
\date{}

\begin{document}

\footnotesize

\subsection*{The Changing Internet}

\noindent \textbf{Networked System:} A cooperating set of autonomous computing devices that exchange data to perform some application goal

\noindent \textbf{Channel constraints:} bound communications speed and reliability

\noindent \textbf{OSI Reference Model:} Application Layer, Presentation Layer, Session Layer, Transport Layer, Network Layer, Data Link Layer, Physical Layer
\textbf{Note:} Real networks don't follow the OSI Reference Model

\noindent \textbf{Physical Layer:} Transmits raw bits over a physical medium.
\textbf{Baseband Data Encoding:}
\textbf{NRZ:} 0 is represented by no change in voltage, 1 is represented by a change in voltage. Easy to miscount bits if long run of same value.
\textbf{Manchester:} Encoding: 0 is represented by a transition from high to low, 1 is represented by a transition from low to high.
\textbf{Modulation:} Allows multiple signals on a channel, modulated onto carriers of different frequency. Amplitude Modulation, Frequency Modulation, Phase Modulation.

\noindent \textbf{Data Link Layer:} provides framing, addressing, media access control, error detection, and flow control.
\textbf{Framing:} Separate the bitstream into meaningful frames of data.
\textbf{Media Access Control:} How devices share the channel. If another transmission is active, the device must wait until the channel is free.


\noindent \textbf{Network Layer:} provides routing, addressing, and packet switching. Internet Protocol (IP).
\textbf{IPv4:} 32-bit address space. Fragmentation difficult at high data rates.
\textbf{IPv6:} 128-bit address space. No in-network fragmentation. Simple header format.
\textbf{Routing:} Each network administered separately - an autonomous system (AS), different technologies and policies.
\textbf{Inter-domain Rounting:} Route advertisements are sent to the routing table of the destination. Border Gateway Protocol (BGP). Advertisements have AS-path.


\noindent \textbf{Transport Layer:} provides end-to-end error recovery, flow control, and multiplexing.
\textbf{TCP:} Connection-oriented, reliable, in-order delivery, flow control, congestion control.
\textbf{UDP:} Connectionless, unreliable, out-of-order delivery, no flow control, no congestion control.

\noindent \textbf{Session Layer:} provides session establishment, maintenance, and termination.
\textbf{Managing Connections:} How to find participants in a connection, how to setup and manage the connection.

\noindent \textbf{Presentation Layer:} provides data representation and encryption.

\noindent \textbf{Application Layer:} provides the interface to the application. Deliver email, stream video, etc.
\textbf{Happy Eyeballs}: The process of trying multiple connections to a server to find one that is available.


\subsection*{Connection Establishment in a Fragmented Network}

% 2a
\noindent \textbf{TCP} is a transport layer protocol, provides a reliable ordered byte stream service over the
best-effort IP network. Provides congestion control.
TCP segments carried as data in IP packets.
IP packets carried as data in link layer frames. Link layer frames delivered over physical layer.
Lost packets are retransmitted, ordering is preserved, message boundaries are not preserved.
TCP follows a client-server model.
The server calls the accept () function to accept incoming connections, while the client initiates a connection by calling connect ().
Calls to send () and recv () are used to send and receive data.
As RTT increases, benefits of increasing bandwidth reduce

% 2b
\noindent \textbf{Impact of TLS:}
HTTP sends and retrives data immediatly once the TCP connection is open.
HTTPS opens a TCP connection, then negotiates secure parameters using TLS.\@
TLS v1.3: extra 1RTT, TLS v1.2: 2RTT.\@
\noindent \textbf{Impact of Ipv6 and dual stack deployments:}
Hosts support a combination of IPv4 and IPv6.

% 2c
\noindent \textbf{Peer-to-peer Connections}
You should be able to run a TCP server on any device, and TCP, UDP based peer-to-peer applications.
Peer-to-peer connetion establishment is difficult due to network address translation (NAT).
\textbf{NAT} is a work around for the shortage of IPv4 addresses, it allows several devices to share a single public IP address.
ISP assigns new range of IP addressses to customer.
Records the mapping, so the reverse changes can be made to any incoming replies as they traverse the NAT in the reverse direction

% 2d
\noindent \textbf{NAT Breaks Applications}:
Client-server applications with server behind NAT fail – need explicit port forwarding
Hard-coding IP addresses, rather than DNS names, in configuration files and application is a bad idea
Outgoing connections create state in NAT, so replies can be translated to reach the correct host on the private network.
No state for incoming connections.
UDP not connection-oriented; NAT can’t detect the end of a flow, so use short timeout to cleanup state once UDP flow has stopped

% 2e
\noindent \textbf{NAT Traversal and Peer-to-Peer Connections Establishment}:
Incoming connections fail, since NAT cannot know how to translate the incoming packets
Peer-to-peer connections can succeed if both NATs think a client server connection is being opened, and the response is coming from the server
Peers connect to referral server on public network, use server to discover the NAT bindings: binding discovery.
Exchange candidate addresses with peer via the referral server: address discovery.
Peers systematically \textbf{probe connectivity}, try to estrablish a connection using every possible combination of candidate addresses.
\textbf{NAT binding discovery}: Requesting a server to tell you the public IP address and port number that you're on.
\textbf{Candidate Exchange}: Each host discovers its candiate IP addresses/port. Peers exchange candidate addresses.
They make TCP connections to the relay server and exchange data over those connections, to reduce latency and to preserve privacy.
\textbf{The ICE algorithm}: try \texttt{connect()} with each candidate address, until a connection is established.
Connection requests sent from a host that passes through a NAT will open a binding that allows a response, even if the connection fails.


\subsection*{Secure Communications}

% 3a
\textbf{The Need for Secure Communications}. Numerous organisations monitor network traffic.
Mechanisms that protect privacy against malicious attackers will also prevent monitoring.
Preventing protocol ossification: network operators deploy middleboxes to monitor or modify traffic.
These middleboxes must inderstand the protocols, this creates ossification.
The more of a protocol that is encrypted, the easier it is to change.

% 3b
\noindent \textbf{Principles of Secure Communication}. Avoid eavesdropping, tampering, and spoofing.
Use encryption to make data useless if intercepted.
\textbf{Symmetric Cryptography}: Same key used for encryption and decryption. Very fast, suitable for bulk data.
\textbf{Public-Key Cryptography}: Different keys for encryption and decryption. Very slow.
\textbf{Hybrid Cryptography}: combines the strengths of public-key and symmetric encryption:
a small symmetric key is securely shared using slower public-key encryption,
and then that key is used for fast, secure communication using symmetric encryption.
This approach ensures both confidentiality and performance.
\textbf{Digital Signatures}: Sender generates a digital signature,
sender calcualte the cryptographic hash of the message,
sender encrypts the hash with their private key.
message and its digital signature are sent to the receiver using hybrid encryption.
Reciever decrypts the message,
reciever calcualtes the cryptographic hash of the message,
reciever decrypts the digitalsignature using the senders public key,
if the signature is valid and the message is unchanged, the reciever can be sure that the message was sent by the claimed sender.
\textbf{Public key infrastructure (PKI)} verifies digital signatures.


% 3c
\noindent \textbf{Transport Layer Security (TLS)} is used to encrypt and authenticate data carried within a TCP connection.
TLS handshake runs within a TCP connection. ClientHello is sent with the ACK, ServerHello is sent in response, Finished message concludes.
ClientHello signals TLS v1.2 but header indicates TLS v1.3, provides the cryptographic algorithms the client supports,
provides the name of the server to which the client is connecting, no data.
ServerHello provides the cryptographic algorithm the server supports, from the ones the client supports,
provides the servers public key and digital signature.
Finished provides clients public key.
TLS record protocol: splits the data into records ($< 2^{14}$ bytes), encrypts each record with a record layer key.
Does not provide record boundaries.
If a client and server have previously communicated, they can re-use a key (0-RTT).
\textbf{Limitations of TLS}: does not encrypt server name, operates within TCP connection, relies on a PKI to validate public keys.


\subsection*{Improving Secure Connection Establishment}

% 4a
\noindent \textbf{Limitations of TLS}.
Connection establishment is still relatively slow, and so first data is sent 2x RTT after the start.
Connection establishment leaks potentially sensitive metadata
The protocol is ossified due to middlebox interference
\textbf{0-RTT Connection Reestablishment}: reuse a preshared key agreed in previous TLS session.
Server sends a PreSharedKey with a SessionTicket to identify the key
When reestablishing a connection: Client sends SessionTicket, data encrypted using corresponding
PreSharedKey, along with ClientHello. The server uses SessionTicket to find saved PreSharedKey,
decrypt the data.
0-RTT data sent with ClientHello using a PreSharedKey is not forward secret.
If a session where PreSharedKey is distributed is compromised,
0-RTT data sent using that key in future connections will also be compromised
\textbf{TLS Metadata Leakage}: server name and PreSharedKey are not encrypted
\textbf{TLS Protocol Ossification}: Some TLS servers fail if ClientHello uses unexpected version number.
TLS 1.3 says its using TLS 1.2, but it says in the header that it is using TLS 1.3.
\textbf{How to Avoid Protocol Ossification}:
\textbf{GREASE}: Generate random extensions and sustain extensibility: send random extensions that are ignored.

% 4b
\noindent \textbf{QUIC Transport Protocol}:
What’s wrong with TLS v1.3 over TCP?,
Slow to connect – due to sequential TCP and TLS handshakes,
Leaks some metadata,
Ossified and hard to extend
QUIC aims to replace TLS v1.3 and TCP with a single secure transport protocol,
Reduce latency by overlapping TLS and transport handshake,
Avoid metadata leakage via pervasive encryption,
Avoid ossification via systematic application of GREASE and encryption
QUIC replaces TCP, TLS, and parts of HTTP
QUIC sends and receives streams of data within a connection.
Up to 262 different streams in each direction in a single QUIC connection
A connection comprises QUIC packets sent within UDP datagrams
\textbf{QUIC Headers}. QUIC packets can be long header packets or short header packets.
Long header packets are used to establish a connection.
Four different long-header packet types, denoted
by the TT field in the header:,
Initial – initiates connection, starts TLS handshake,
0-RTT – idempotent data sent with initial handshake, when resuming a session,
Handshake – completes connection establishment,
Retry – used to force address validation
The is one short header packet defined in QUIC:
1-RTT – Used for all packets sent after the TLS handshake is complete
QUIC packet contain an encrypted sequence of frames

% 4c
\noindent \textbf{QUIC Connection Establishment}. A QUIC connection proceeds in two phases: handshake and data transfer.
\textbf{C} $\rightarrow$ \textbf{S}: QUIC Initial packet
\textbf{S} $\rightarrow$ \textbf{C}: QUIC Initial and Handshake Packet
\textbf{C} $\rightarrow$ \textbf{S}: QUIC Initial, Handshake, and 1-RTT Packet
Initial packet contains a CRYPTO frame that contains TLS ClientHello or ServerHello, also synchronises the client and server state.
QUIC Initial packets also carry optional Token Server can refuse the initial connection attempt,
and send a Retry packet containing a Token.
Client must then retry the connection, providing the Token in its Initial packet
QUIC supports TLS 0-RTT session re-establishment: QUIC Initial packet contains CRYPTO frame with a TLS ClientHello
and a SessionTicket QUIC 0-RTT packet included in the same UDP datagram contains a STREAM frame carrying idempotent 0-RTT data:
\textbf{Data Transfer}. After handshake has finished, QUIC switches to sending short header packets
The short header contains a Packet Number field, Packet numbers increase by one for each packet sent.
ACK frames indicate received packet numbers. QUIC packet numbers count packets sent; TCP sequence numbers count bytes of data sent.
QUIC never retransmits packets – retransmits frames sent in lost packets in new packets, with new packet numbers
QUIC sends acknowledgements of received packets in ACK frames.
Sent inside a long- or short-header packets; unlike TCP, not part of headers.
Indicate sequence numbers of QUIC packets that were received, not frames
Data is sent within STREAM frames, sent within QUIC packets, contains a stream identifier, offset of the data, data length and data.

% 4d
\noindent \textbf{QUIC over UDP}. Why run QUIC over UDP? To ease end-system deployment.
To work around protocol ossification.
Entire packet except invariant fields and the last 7-bits of the first byte is encrypted.
QUIC authenticates all data.
\textbf{Why is QUIC desirable?}.
Reduces secure connection establishment latency.
Reduces risk of ossification; easy to deploy.
Supports multiple streams within a single connection.
\textbf{Why is QUIC problematic?}.
Libraries and support new, poorly documented, and frequently buggy.
CPU usage is high compared to TLS-over-TCP.

\subsection*{Reliability and Data Transfer}

% 5a
\noindent \textbf{Packet Loss in the Internet}.
The Internet is a best effort packet delivery network – it is unreliable.
IP packets may be lost, delayed, reordered, or corrupted in transit.
Only put functionality that is absolutely necessary in the network, leave everything else to end systems.
If a connection is to be reliable, it cannot guarantee timeliness.
If a connection is to be timely, it cannot guarantee reliability.

% 5b
\noindent \textbf{Unreliable Data Using UDP}.
The \texttt{sendto ()} call sends a single datagram.
Each call to \texttt{sendto ()} can send to a different address, even though they use the same socket.
The \texttt{recv ()} call may be used to read a single datagram, but doesn’t provide the source address of the datagram.
Most code uses \texttt{recvfrom ()} instead – this fills in the source address of the received datagram.
UDP does not attempt to provide sequencing, reliability, or timing recovery.

% 5c
\noindent \textbf{Reliable Data Using TCP}.
TCP ensures data delivered reliably and in order.
TCP sends acknowledgements for segments as they are received; retransmits lost data.
TCP will recover original transmission order if segments are delayed and arrive out of order.
\textbf{TCP Segments and Sequence Numbers}. Data is split into segments, with a sequence number, each placed in a TCP packet.
Sequence numbers start from the value sent in the TCP handshake.
Segments sent to acknowledge each received segment – contains acknowledgment number indicating sequence number of the next
contiguous byte expected.
Can send delayed acknowledgements if there is no data to send in reverse direction.





\end{document}
