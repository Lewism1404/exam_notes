\documentclass{article}
\usepackage[utf8]{inputenc}
\usepackage{amsmath}
\usepackage{amsthm}
\usepackage[margin=2px]{geometry}

\newtheorem{definition}{Definition}
\newtheorem{note}{Note}

\title{Networked Systems}
\author{}
\date{}

\begin{document}

\footnotesize

\subsection*{The changing Internet}

\begin{definition}
    A Networked System is a cooperating set of autonomous computing devices that exchange data to perform some application goal
\end{definition}
\vspace{-15pt}
\begin{note}
    Channel constraints bound communications speed and reliability
\end{note}
\vspace{-15pt}
\begin{definition}
    The OSI Reference Model: Application Layer, Presentation Layer, Session Layer, Transport Layer, Network Layer, Data Link Layer, Physical Layer
\end{definition}
\vspace{-15pt}
\begin{note}
    Real networks don't follow the OSI Reference Model
\end{note}
\vspace{-8pt}

\noindent \textbf{Physical Layer:} Transmits raw bits over a physical medium. 

\noindent Baseband Data Encoding:
\noindent NRZ: 0 is represented by no change in voltage, 1 is represented by a change in voltage. Easy to miscount bits if long run of same value.
\noindent Manchester: Encoding: 0 is represented by a transition from high to low, 1 is represented by a transition from low to high.

\noindent Modulation: Allows multiple signals on a channel, modulated onto carriers of different frequency. Amplitude Modulation, Frequency Modulation, Phase Modulation.


\noindent \textbf{Data Link Layer:} provides framing, addressing, media access control, error detection, and flow control.

\noindent Framing: Separate the bitstream into meaningful frames of data.

\noindent Media Access Control: How devices share the channel. If another transmission is active, the device must wait until the channel is free.


\noindent \textbf{Network Layer:} provides routing, addressing, and packet switching. Internet Protocol (IP).

\noindent IPv4: 32-bit address space. Fragmentation difficult at high data rates.

\noindent IPv6: 128-bit address space. No in-network fragmentation. Simple header format.

\noindent Routing: Each network administered separately - an autonomous system (AS), different technologies and policies.
wh
\noindent Inter-domain Rounting: Route advertisements are sent to the routing table of the destination. Border Gateway Protocol (BGP). Advertisements have AS-path.


\noindent \textbf{Transport Layer:} provides end-to-end error recovery, flow control, and multiplexing.

\noindent TCP: Connection-oriented, reliable, in-order delivery, flow control, congestion control.

\noindent UDP: Connectionless, unreliable, out-of-order delivery, no flow control, no congestion control.


\noindent \textbf{Session Layer:} provides session establishment, maintenance, and termination.

\noindent Managing Connections: How to find participants in a connection, how to setup and manage the connection.

\noindent \textbf{Presentation Layer:} provides data representation and encryption.

\noindent \textbf{Application Layer:} provides the interface to the application. Deliver email, stream video, etc.

\noindent \textbf{Happy Eyeballs}: The process of trying multiple connections to a server to find one that is available.


\end{document}