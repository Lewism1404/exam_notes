\documentclass{article}
\usepackage[utf8]{inputenc}
\usepackage{amsmath}
\usepackage{amsthm}
\usepackage[margin=2px]{geometry}

\newtheorem{definition}{Definition}
\newtheorem{note}{Note}

\title{Networked Systems}
\author{}
\date{}

\begin{document}

\footnotesize

\subsection*{The changing Internet}

\noindent \textbf{Networked System:} A cooperating set of autonomous computing devices that exchange data to perform some application goal

\noindent \textbf{Channel constraints:} bound communications speed and reliability

\noindent \textbf{OSI Reference Model:} Application Layer, Presentation Layer, Session Layer, Transport Layer, Network Layer, Data Link Layer, Physical Layer
\textbf{Note:} Real networks don't follow the OSI Reference Model

\noindent \textbf{Physical Layer:} Transmits raw bits over a physical medium. 
\textbf{Baseband Data Encoding:}
\textbf{NRZ:} 0 is represented by no change in voltage, 1 is represented by a change in voltage. Easy to miscount bits if long run of same value.
\textbf{Manchester:} Encoding: 0 is represented by a transition from high to low, 1 is represented by a transition from low to high.
\textbf{Modulation:} Allows multiple signals on a channel, modulated onto carriers of different frequency. Amplitude Modulation, Frequency Modulation, Phase Modulation.

\noindent \textbf{Data Link Layer:} provides framing, addressing, media access control, error detection, and flow control.
\textbf{Framing:} Separate the bitstream into meaningful frames of data.
\textbf{Media Access Control:} How devices share the channel. If another transmission is active, the device must wait until the channel is free.


\noindent \textbf{Network Layer:} provides routing, addressing, and packet switching. Internet Protocol (IP).
\textbf{IPv4:} 32-bit address space. Fragmentation difficult at high data rates.
\textbf{IPv6:} 128-bit address space. No in-network fragmentation. Simple header format.
\textbf{Routing:} Each network administered separately - an autonomous system (AS), different technologies and policies.
\textbf{Inter-domain Rounting:} Route advertisements are sent to the routing table of the destination. Border Gateway Protocol (BGP). Advertisements have AS-path.


\noindent \textbf{Transport Layer:} provides end-to-end error recovery, flow control, and multiplexing.
\textbf{TCP:} Connection-oriented, reliable, in-order delivery, flow control, congestion control.
\textbf{UDP:} Connectionless, unreliable, out-of-order delivery, no flow control, no congestion control.

\noindent \textbf{Session Layer:} provides session establishment, maintenance, and termination.
\textbf{Managing Connections:} How to find participants in a connection, how to setup and manage the connection.

\noindent \textbf{Presentation Layer:} provides data representation and encryption.

\noindent \textbf{Application Layer:} provides the interface to the application. Deliver email, stream video, etc.
\textbf{Happy Eyeballs}: The process of trying multiple connections to a server to find one that is available.


\subsection*{Connection Establishment in a Fragmented Network}

% 2a
\noindent \textbf{TCP} is a transport layer protocol, provides a reliable ordered byte stream service over the 
best-effort IP network. Provides congestion control. 
TCP segments carried as data in IP packets. 
IP packets carried as data in link layer frames. Link layer frames delivered over physical layer.
Lost packets are retransmitted, ordering is preserved, message boundaries are not preserved.
TCP follows a client-server model. 
The server calls the accept () function to accept incoming connections, while the client initiates a connection by calling connect ().
Calls to send () and recv () are used to send and receive data.
As RTT increases, benefits of increasing bandwidth reduce

% 2b
\noindent \textbf{Impact of TLS:}
HTTP sends and retrives data immediatly once the TCP connection is open.
HTTPS opens a TCP connection, then negotiates secure parameters using TLS.\@
TLS v1.3: extra 1RTT, TLS v1.2: 2RTT.\@
\noindent \textbf{Impact of Ipv6 and dual stack deployments:}
Hosts support a combination of IPv4 and IPv6.

% 2c
\noindent \textbf{Peer-to-peer Connections}
You should be able to run a TCP server on any device, and TCP, UDP based peer-to-peer applications.
Peer-to-peer connetion establishment is difficult due to network address translation (NAT).
\textbf{NAT} is a work around for the shortage of IPv4 addresses, it allows several devices to share a single public IP address.
ISP assigns new range of IP addressses to customer.
Records the mapping, so the reverse changes can be made to any incoming replies as they traverse the NAT in the reverse direction

% 2d
\noindent \textbf{NAT Breaks Applications}:
Client-server applications with server behind NAT fail – need explicit port forwarding
Hard-coding IP addresses, rather than DNS names, in configuration files and application is a bad idea
Outgoing connections create state in NAT, so replies can be translated to reach the correct host on the private network.
No state for incoming connections.
UDP not connection-oriented; NAT can’t detect the end of a flow, so use short timeout to cleanup state once UDP flow has stopped

% 2e
\noindent \textbf{NAT Traversal and Peer-to-Peer Connections Establishment}:
Incoming connections fail, since NAT cannot know how to translate the incoming packets
Peer-to-peer connections can succeed if both NATs think a client server connection is being opened, and the response is coming from the server
Peers connect to referral server on public network, use server to discover the NAT bindings: binding discovery.
Exchange candidate addresses with peer via the referral server: address discovery.
Peers systematically \textbf{probe connectivity}, try to estrablish a connection using every possible combination of candidate addresses.
\textbf{NAT binding discovery}: Requesting a server to tell you the public IP address and port number that you're on.
\textbf{Candidate Exchange}: Each host discovers its candiate IP addresses/port. Peers exchange candidate addresses.
They make TCP connections to the relay server and exchange data over those connections, to reduce latency and to preserve privacy.
\textbf{The ICE algorithm}: try \texttt{connect()} with each candidate address, until a connection is established.
Connection requests sent from a host that passes through a NAT will open a binding that allows a response, even if the connection fails.


\subsection*{Secure Communications}

% 3a
\textbf{The Need for Secure Communications}. Numerous organisations monitor network traffic. 
Mechanisms that protect privacy against malicious attackers will also prevent monitoring.
Preventing protocol ossification: network operators deploy middleboxes to monitor or modify traffic. 
These middleboxes must inderstand the protocols, this creates ossification. 
The more of a protocol that is encrypted, the easier it is to change.

% 3b
\noindent \textbf{Principles of Secure Communication}. Avoid eavesdropping, tampering, and spoofing.
Use encryption to make data useless if intercepted.
\textbf{Symmetric Cryptography}: Same key used for encryption and decryption. Very fast, suitable for bulk data.
\textbf{Public-Key Cryptography}: Different keys for encryption and decryption. Very slow.
\textbf{Hybrid Cryptography}: combines the strengths of public-key and symmetric encryption: 
a small symmetric key is securely shared using slower public-key encryption, 
and then that key is used for fast, secure communication using symmetric encryption. 
This approach ensures both confidentiality and performance.





\end{document}
