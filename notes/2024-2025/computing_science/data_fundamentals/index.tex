\documentclass{article}
\usepackage[utf8]{inputenc}
\usepackage{amsmath}
\usepackage{amssymb}
\usepackage[margin=2px]{geometry}
\title{Data Fundamentals}
\author{}
\date{}

\begin{document}

\small

\subsection*{Unit 1: Arrays (p.~2)}
\begin{itemize}
    \item Basic datatypes (lists, strings, dictionaries, etc.)
    \item Why and where we use arrays (images, scientific data, vectorization)
    \item Array shapes (vectors, matrices, tensors) and axes
    \item Creating arrays (zeros, ones, random, ranges)
    \item Slicing, indexing, rearranging, and Boolean masking
    \item Array operations (arithmetic, aggregate functions, broadcasting)
\end{itemize}

\subsection*{Unit 2: Numerical Issues and Tensors (p.~9)}
\begin{itemize}
    \item Floating-point representation (sign, exponent, mantissa)
    \item Special values: infinities, NaNs; causes and behavior
    \item Round-off, precision, overflow, underflow, and stability issues
    \item Memory layout of arrays (strides, transpose, reshaping)
    \item Tensors: operations and vectorized computation strategies
\end{itemize}

\subsection*{Unit 3: Scientific Visualisation (p.~16)}
\begin{itemize}
    \item Fundamentals of data plotting (grammar of graphics, guides, geoms)
    \item 1D and 2D plots, histograms, scatterplots, contour plots
    \item Choosing appropriate scales (linear, log, polar) and facets
    \item Statistical transforms: binning, smoothing, regression
    \item Plotting best practices for clarity, labels, and communicating uncertainty
\end{itemize}

\subsection*{Unit 4: Vector Spaces and Matrices (p.~23)}
\begin{itemize}
    \item Vectors in $\mathbb{R}^n$, norms, and inner products
    \item Matrix basics: multiplication, transposition, linear operators
    \item Matrices as transformations, adjacency matrices for graphs
    \item Covariance matrices and their role in data analysis
\end{itemize}

\subsection*{Unit 5: Computational Linear Algebra (p.~30)}
\begin{itemize}
    \item Matrix decompositions (eigendecomposition, SVD)
    \item Eigenvalues, eigenvectors, and stability analysis
    \item Determinant, trace, and special matrix properties (positive/negative definiteness)
    \item Inversion, pseudo-inverse, condition numbers, and numerical issues
    \item Low-rank approximations and dimensionality reduction
\end{itemize}

\subsection*{Unit 6: Introduction to Optimization (p.~37)}
\begin{itemize}
    \item Formulating optimization problems (parameters, objective functions)
    \item Types of optimization (continuous, discrete, constrained)
    \item Convexity vs.\ non-convexity, local/global minima
    \item Simple algorithms (grid search, random search) and basic iterative methods
\end{itemize}

\subsection*{Unit 7: Numerical Nonlinear Optimization (p.~46)}
\begin{itemize}
    \item Gradient-based methods (gradient descent, Newton’s method)
    \item Convergence criteria and step-size strategies
    \item Handling constraints (Lagrange multipliers, penalty methods)
    \item Example applications to real-world nonlinear problems
\end{itemize}

\subsection*{Unit 8: Probability and Random Variables (p.~53)}
\begin{itemize}
    \item Foundations of probability (axioms, conditional probability)
    \item Random variables, distributions (discrete and continuous)
    \item Expectation, variance, common distributions (Normal, Poisson, etc.)
    \item Transformations and basic limit theorems
\end{itemize}

\subsection*{Unit 9: Sampling and Inference (p.~57)}
\begin{itemize}
    \item Sampling methods (Monte Carlo, importance sampling)
    \item Parameter estimation (maximum likelihood, Bayesian inference)
    \item Hypothesis testing and confidence intervals
    \item Practical considerations in computational inference
\end{itemize}

\subsection*{Unit 10: Time Series and Signals (p.~62)}
\begin{itemize}
    \item Basic concepts (stationarity, autocorrelation, power spectra)
    \item Discrete Fourier transform, filtering, convolution
    \item Signal processing techniques and applications
    \item Introduction to forecasting and seasonality in time series
\end{itemize}

\end{document}
